\section{DESAFIO - Parte 2}

O ambiente deve ser todo configurado através de gerenciador de configuração, o que deverá ser entregue é um repositório git contendo os arquivos de configuração que serão aplicados em uma máquina virtual "zerada". Caso necessário descrever como executar o processo de aplicação da configuração na máquina virtual. Ao final da tarefa e execução do processo, deveremos ter um ambiente funcional;

- É recomendado que o repositório git seja entregue com commits parciais, mostrando a evolução de seu código e pensamento. Caso prefira nos informe um url de git público ou então compacte todos os arquivos em um .tar.gz mantendo a pasta .git em conjunto;

- No ambiente deverá estar rodando uma aplicação node.js de exemplo,
conforme código abaixo. A versão do node.js deverá ser a última versão
LTS disponível em: https://nodejs.org/en/download/. A aplicação node
abaixo possui a dependência da biblioteca express. Garanta que seu
processo de bootstrap instale essa dependência ( última versão estável
disponível em: http://expressjs.com/ ) e rode o processo node em
background. De uma forma dinâmica garanta que seja criado uma
instância node para cada processador existente na máquina ( a máquina
poderá ter de 1 a 32 processadores );

Construa dentro de sua automação um processo de deploy e rollback seguro e rápido da aplicação node. O deploy e rollback deverá garantir a instalação das dependências node novas (caso sejam adicionadas ou alteradas a versão de algum dependência por exemplo), deverá salvar a versão antiga para possível rollback e reiniciar todos processos node sem afetar a disponibilidade global da aplicação na máquina;

- A aplicação Node deverá ser acessado através de um Servidor Web configurado como Proxy Reverso e que deverá intermediar as conexões HTTP e HTTPS com os clientes e processos node. Um algoritmo de balanceamento deve ser configurado para distribuir entre os N processos node a carga recebida;

- A fim de garantir a disponibilidade do serviço, deverá estar funcional uma monitoração do processo Node e Web para caso de falha, o mesmo deve reiniciar ou subir novamente os serviços em caso de anomalia;

- Desenvolva um pequeno script que rode um teste de carga e demostre qual o Throughput máximo que seu servidor consegue atingir;

- Desenvolva um script que parseie o log de acesso do servidor Web e deverá rodar diariamente e enviar por e-mail um simples relatório, com a frequência das requisições e o respectivo código de resposta (ex:5 /index.html 200);

- Por fim; rode o seu parser de log para os logs gerados pelo teste de carga, garantindo que seu script terá performance mesmo em casos de logs com milhares de acessos;