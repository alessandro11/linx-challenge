\section{DESAFIO - Parte 1} \label{sec:desafio1}

Estamos enviando pelos links abaixo alguns binários executáveis (ELF
64-bit LSB) que realizam tarefas bem simples, que podem ou não ser
úteis. O exercício é que você descubra o que esses binários fazem,
utilizando as ferramentas que julgar mais adequadas. Como resposta,
esperamos que você nos diga o que você acha que eles fazem e quais
foram as ferramentas usadas para isso, bem como uma linha geral do seu
raciocínio para chegar às conclusões.\\
<<<<<<< HEAD

\par Nesta seção, para encontrar as soluções dos problemas propostos, foi
utilizado primordialmente as seguintes ferramentas:

=======

\par Nesta seção, para encontrar as soluções dos problemas propostos, foi
utilizado primordialmente as seguintes ferramentas:

>>>>>>> bin-ddb1c9
\begin{verbatim}
GNU objdump (GNU Binutils) 2.33.1
Copyright (C) 2019 Free Software Foundation, Inc.
This program is free software; you may redistribute it under the terms of
the GNU General Public License version 3 or (at your option) any later version.
This program has absolutely no warranty.
\end{verbatim}

Esta ferramenta foi utilizada para obter o desmontagem (disassembly)
dos binários, bem como fazer uma analise das seções e símbolos dos
binários. Desta forma, possibilitando uma leitura do código montado
(assembly).

Os seguintes comandos foram utilizados para analise das seções e ou
cabeçalhos:

Para analises de cabeçalhos e seções\\
$$objdump \ \ -x \ \ <nome\_do\_arquivo>$$
$$objdump \ \ -T \ \ <nome\_do\_arquivo>$$
$$objdump \ \ -f \ \ <nome\_do\_arquivo>$$

Para disassembly:\\
$$objdump \ \ -S \ \ <nome\_do\_arquivo>$$
$$objdump \ \ -D \ \ <nome\_do\_arquivo>$$
$$objdump \ \ -d \ \ <nome\_do\_arquivo>$$

\begin{verbatim}
GNU gdb (GDB) 8.3.1
Copyright (C) 2019 Free Software Foundation, Inc.
License GPLv3+: GNU GPL version 3 or later <http://gnu.org/licenses/gpl.html>
This is free software: you are free to change and redistribute it.
There is NO WARRANTY, to the extent permitted by law.
\end{verbatim}

Esta possibilitou a confirmação de qualquer conjectura obtida pela
ferramenta anterior, executando o assembly passo a passo.

Sua execução com os seguintes parâmetros:
$$gdb \ \ -tui \ \ <programa>$$
<<<<<<< HEAD

Os binários e seus respectivos disassembly, um pequeno leia-me podem
ser encontrados no repositório:\\

\href{https://github.com/alessandro11/desafio-1}{https://github.com/alessandro11/desafio-1}


\subsection{Salvando entrada do usuário em /tmp/<USER>}
\noindent Binário: \href{https://s3.amazonaws.com/chaordic-desafio-cloud/cc9621}{cc9621}\\
Disassembly: \href{https://github.com/alessandro11/desafio-1/blob/master/cc9621.as}{https://github.com/alessandro11/desafio-1/blob/master/cc9621.as}

=======

Os binários e seus respectivos disassembly, um pequeno leia-me podem
ser encontrados no repositório:\\

\href{https://github.com/alessandro11/desafio-1}{https://github.com/alessandro11/desafio-1}


\subsection{Salvando entrada do usuário em /tmp/<USER>}
Binário: \href{https://s3.amazonaws.com/chaordic-desafio-cloud/cc9621}{cc9621}
Disassembly:
\href{https://github.com/alessandro11/desafio-1/blob/master/cc9621.as}{https://github.com/alessandro11/desafio-1/blob/master/cc9621.as}

>>>>>>> bin-ddb1c9
Este programa abre um arquivo no diretório ``/tmp/'', sendo nome do
arquivo o usuário obtido do ambiente de execução do programa. Espera a
entrada do usuário, quando finalizado com um \emph{Enter}, o
dado entrado é salvo no arquivo ``/tmp/USER''.

\lstinputlisting[caption = {Escreve entrada do usuário em ``/tmp/USER''.},label={prog:cc9621}]{cc9621.as}

No Assembly~\ref{prog:cc9621} é mostrado o \emph{main} do programa, as
outras seções foram truncadas, assim focamos apenas no núcleo de que o
programa faz (não irei abordar cada instrução, apenas as
relevantes). Também é apresentado como comentários em inglês mais
detalhes das instruções relevantes.

A primeira instrução relevante é 4007\{48,52\}, após a linha 15. No qual
armazena a string ``/tmp/'' em uma variável na pilha. A instrução
400770 preenche com zeros um buffer de tamanho 24, indicado pela
constante armazenado em \emph{RCX}, no qual controla quantas vezes será
repetido a instrução \emph{stos}. Este buffer será usado pelo
\emph(scanf). Na instrução 400778 é obtido o usuário que está
executando o programa, em meu caso o usuário é ``m3cool''. A 400798
concatena a string ``/tmp/'' com ``m3cool''. Na 4007b1 é feito a
chamada ao \emph{scanf} para obter a entrada do usuário, uma string
qualquer. A 4007c5 abre o arquivo no caminho ``/tmp/m3cool''. Na
4007ef (\emph(fputs) escreve a string obtida pelo \emph{scanf} no
arquivo aberto. Na 4007fe fecha o arquivo e o programa é encerrado.


\subsection{Conway's Game of Life}
<<<<<<< HEAD
\noindent Binário: \href{https://s3.amazonaws.com/chaordic-desafio-cloud/d3ea79}{d3ea79}\\
=======
Binário: \href{https://s3.amazonaws.com/chaordic-desafio-cloud/d3ea79}{d3ea79}
>>>>>>> bin-ddb1c9
Disassembly: \href{https://github.com/alessandro11/desafio-1/blob/master/d3ea79.as}{https://github.com/alessandro11/desafio-1/blob/master/d3ea79.as}

Este programa gera a saída do jogo Conway's Game of Life,
escrevendo-o no arquivo ``/tmp/<USER>''. O
link~\footnote{\href{http://en.wikipedia.org/wiki/Conway\%27s\_Game\_of\_Life}{http://en.wikipedia.org/wiki/Conway\%27s\_Game\_of\_Life}}
para a Wiki do jogo também é escrito no arquivo. A saída gerada na
execução deste programa está no repositório em
\href{https://github.com/alessandro11/desafio-1/blob/master/m3cool}{/tmp/m3cool}.

O disassembly parcial do binário pode ser observado nas rotinas abaixo:

\lstinputlisting[caption = {Remove e recria o arquivo ``/tmp/<USER>''
com o Conway's Game of Life},label={prog:d3ea79-main}]{d3ea79-main.as}

No Assembly~\ref{prog:d3ea79-main} observamos as respectivas chamadas:
\emph{clear}, \emph{welcome} \emph{play}. Iremos abordar cada uma das rotinas.\\

\par\textbf{\emph{clear}}:
\lstinputlisting[caption = {Remove arquivo
 ``/tmp/<USER>''},label={prog:d3ea79-clear}]{d3ea79-clear.as}

A instrução 400967 atribui ao registrador \emph{RAX}
a string ``/tmp/''. Na instrução 400992 é atribuído o ponteiro para a
string (constante) ``USER'', no qual é passado como parâmetro para
\emph{getenv}. Este retorna o valor da variável de ambiente <USER>.
A instrução 4009bc concatena a string que estava em \emph{RAX} e valor da
variável de ambiente. No meu caso, resultando na string
``/tmp/m3cool''. A 4009cb remove o arquivo ``/tmp/m3cool''.\\

\par\textbf{\emph{welcome}}:
\lstinputlisting[caption = {Escreve cabeçalho no arquivo
 ``/tmp/<USER>''},label={prog:d3ea79-welcome}]{d3ea79-welcome.as}

O Assembly~\ref{prog:d3ea79-welcome}, na instrução 400a00 à 400a55
concatena strings como em~\ref{prog:d3ea79-clear}. Instrução 400a69
abre o arquivo (``/tmp/m3cool''). Na instrução 400aa7 carrega ponteiro
para string (constante) ``Welcome to the Game of Life.\textbackslash n'' a
subsequente escreve no arquivo esta string. A instrução 400ab8 carrega a
string (constante) ``http://en.wikipedia.org/wiki/Conway\%27s\_Game\_of\_Life'' e
subsequente escreve no arquivo e então fecha-o.\\

\par\textbf{\emph{play}}, \textbf{\emph{print}},
\textbf{\emph{evolve}} são rotinas responsáveis por gerar o jogo.


\subsection{Fork()}
<<<<<<< HEAD
\noindent Binário: \href{https://s3.amazonaws.com/chaordic-desafio-cloud/da87fa}{da87fa}\\
=======
Binário: \href{https://s3.amazonaws.com/chaordic-desafio-cloud/da87fa}{da87fa}
>>>>>>> bin-ddb1c9
Disassembly:
\href{https://github.com/alessandro11/desafio-1/blob/master/d3ea79.as}{https://github.com/alessandro11/desafio-1/blob/master/d3ea79.as}

Este programa fica em um laço infinito, ou até receber um SIGHUP,
executando \emph{fork()} dez vezes. Na décima, é executado um
\emph{sleep} de dez segundos, e torna a executar outros dez
forks. Como é mapeado SIGCHLD para SIGHUP, o processo filho inicia e
encerra.

Analisando a desmontagem do binário no Assembly~\ref{prog:da87fa}:

\lstinputlisting[caption = {Fork do processo},label={prog:da87fa}]{da87fa.as}

As instruções 4006\{16, 1b, 20\} mapeiam o sinal de SIGCHLD para
SIGHUP, o que acarreta na finalização do processo, assim que o processo
filho inicia. Da instrução 40062a à 400653 ocorre o laço dos
forks. Nas instruções 40065\{0, 3\} controla quantos forks serão
feitos enquanto o registrador \emph{EBX} não atinge dez, um if. Quando o
registrador \emph{EBX} atinge dez, então é executado um \emph{sleep} de dez
segundos e volta para o inicio do laço.

O interessante para sair do laço, finalizando o processo, nunca
ocorrerá. Conforme as instruções 40063f executa uma comparação
\emph{test}, no qual somente será igual se o registrador \emph{EAX} for zero,
porém a instrução lê o pid do processo filho que foi salvo na pilha,
atribui este valor ao registrador \emph{EAX} e executa o \emph{test}, logo
não é atribuído pid zero para processos, por definição do SO. Então a
instrução 400641 sempre fará o salto para a verificação se já executou
os dez forks antes do \emph{sleep}, e nunca executará 400648,
finalizar processo.


\subsection{Multi Thread escutando na porta 8011}
\noindent Binário: \href{https://s3.amazonaws.com/chaordic-desafio-cloud/ddb1c9}{ddb1c9}\\
Disassembly:
\href{https://github.com/alessandro11/desafio-1/blob/master/ddb1c9.as}{https://github.com/alessandro11/desafio-1/blob/master/ddb1c9.as}\\

\par Este programa abre três threads e atende requisições http,
cujo programa escuta em localhost na porta 8011. Ao fazer uma
requisição http para \texttt{http://localhost:8011} apenas o caractere
'!' é retornado. A depuração deste código é bem complicado, por ter
sido compilado com a linguagem GO é gerado um backtrace de mais de 20
chamadas, e em paralelo para todas as threads, o resulta em vários
fluxos de execuções. Portando a abordagem de leitura do dissasembly,
como nos anteriores, e depuração via \emph{gdb} não foram
suficientes.

Para triangular o que o programa faz, utilizei mais algumas
ferramentas: \emph{lsof} para identificar quais descritores de
arquivos abertos ele possui, como é mostrado na figura \todo{figura
  lsof ddb1c9}. O programa e seu parâmetros \emph{netstat -natup}
(\emph{ss -latp}, no qual retornam processos e suas respectivas portas
abertas. O \emph{netcat localhost 8011}/\emph{telnet localhost 8011}
para tentar enviar GET, e outros comandos aleatórios. Para identificar
possível comunicação nesta porta utilizei \emph{tcpdump -vv -n -i lo
  port 8011} para monitorar o tráfego de rede.

Como conclusão do funcionamento, reforço o parágrafo um, atende
requisições http retornando '!'.
